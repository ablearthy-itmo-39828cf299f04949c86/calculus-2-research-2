\section{Задача № 2}

Дано векторное поле \(\vec{H} : \left( e^x, -e^y \right)\).

\begin{enumerate}
  \item Убедитесь, что данное векторное поле потенциально.
  \item Найдите уравнения векторных линий. Изобразите векторные линии на
        рисунке.
  \item \label{itm:02-stmt}
    Найдите потенциал поля при помощи криволинейного интеграла.
  \item Найдите уравнения линий уровня потенциала (эквипотенциальных линий).
        Изобразите линии уровня потенциала.
  \item Докажите ортогональность найденных векторных линий поля и линий уровня
        потенциала. Проиллюстрируйте ортогональность на графике.
  \item Выберите какую-либо векторную линию поля и зафиксируйте на ней точки
        \(A\) и \(B\), выбрав для них числовые координаты.
        Вычислите работу поля вдоль этой линии,
        используя найденный в п. \ref{itm:02-stmt} потенциал.
\end{enumerate}

\subsection{Решение}

Обозначим \(\vec{H} = P(x, y) \vec{i} + Q(x, y) \vec{j}\), где
\(P(x, y) = e^x\), \(Q(x, y ) = -e^y\).

Поле будет потенциальным, если найдется функция \(\Phi(x, y)\) (называемая
потенциалом) такая, что её частные производные по \(x\) и \(y\) будут равны
\(P(x, y)\,dx\) и \(Q(x, y)\,dy\) соответственно.

Рассмотрим функцию \(\Phi(x, y) = e^x - e^y\), вычислим её частные производные:

\begin{equation*}\begin{split}
    \Phi'_x = e^x\,dx \\
    \Phi'_y = -e^y\,dy
  \end{split}\end{equation*}

Получаем, что \(\Phi'_x = P(x, y)\,dx\) и \(\Phi'_y = Q(x, y)\,dy\), значит
функция \(\Phi(x, y)\) является потенциалом этого поля, а значит поле является
потенциальным.

\bigskip

Найдем уравнения векторных линий, для этого решим дифференциальное уравнение
\(Q(x, y)\,dx = P(x, y)\,dy\):

\begin{equation*}\begin{split}
    -e^y\,dx = e^x\,dy \\
    \int -e^{-x}\,dx = \int e^{-y}\,dy \\
    e^{-x} = -e^{-y} + A \hspace{3pt} (A > 0, A \in \mathbb{R})\\
    e^{-x} + e^{-y} = A
  \end{split}\end{equation*}

Получили уравнение для векторных линий \(e^{-x} + e^{-y} = A\).

Изобразим на графике некоторые из них при разных значениях \(A\).

\begin{figure}[!htbp]
  \centering
  \asyinclude{asy/02-01.asy}
  \caption{Векторные линии}
\end{figure}

\bigskip

Найдем потенциал \(\Phi\) поля с помощью криволинейного интеграла. Пусть точка
\(O(0, 0)\) это начало координат, а \(M(x, y)\) некоторая произвольная точка,
вычислим следующий криволинейный интеграл:

\begin{equation*}\begin{split}
    \int\limits_{OM} P(x, y)\,dx + Q(x, y)\,dy = \\
    \int\limits_{OM} e^x \,dx - e^y \,dy
  \end{split}\end{equation*}

Т.к. поле потенциально, то данный интеграл не зависит от пути и может быть
вычислен разбиением на два обычных интеграла:

\begin{equation*}\begin{split}
    \int\limits_{OM} e^x \,dx - e^y \,dy = \\
    \int_0^x e^x\,dx - \int_0^y e^y\,dy = \\
    e^x \bigg\vert_0^x - e^y \bigg\vert_0^y = \\
    e^x - 1 - e^y + 1 = \\
    e^x - e^y
  \end{split}\end{equation*}

Таким образом потенциал поля равен \(\Phi = e^x - e^y\).

\bigskip

Эквипотенциальные линии это линии, во всех точках которых потенциал имеет
одинаковое значение. Они будут задаваться уравнением \(\Phi = const\), т.е.
\(e^x - e^y = B \hspace{3pt} (B \in \mathbb{R})\).

Изобразим на графике некоторые из них при разных значениях \(B\).

\begin{figure}[!htbp]
  \centering
  \asyinclude{asy/02-02.asy}
  \caption{Линии уровня потенциала}
\end{figure}

\bigskip

Докажем, что эквипотенциальные линии перпендикулярны векторным линиям поля. Эти
линии будут перпендикулярны, если перпендикулярны их касательные в точке
пересечения. Обозначим точку их пересечения \((x_0, y_0)\).

Найдем наклон касательной кривой \(e^{-x} + e^{-y} - A = 0\) в этой точке.
Наклон касательной в точке равен производной в этой точке, для вычисления
производной воспользуемся формулой
\(F(x, y) = 0 \implies y' = -\dfrac{F'_x}{F'_y}\). Найдем частные производные
функции \(F(x, y)\):

\begin{equation*}\begin{split}
    F'_x = e^{-x} \cdot (-1) = -e^{-x} \\
    F'_y = e^{-y} \cdot (-1) = -e^{-y}
  \end{split}\end{equation*}

Таким образом наклон касательной к графику \(e^{-x} + e^{-y} = A\) в точке
\((x_0, y_0)\) будет равен
\(-\dfrac{e^{-x_0}}{e^{-y_0}} = -\dfrac{e^{y_0}}{e^{x_0}}\).

Аналогично найдем наклон касательной к кривой \(G(x, y) = e^x - e^y - B = 0\) в
той же точке \((x_0, y_0)\):

\begin{equation*}\begin{split}
    G'_x = e^x \\
    G'_y = -e^y
  \end{split}\end{equation*}

Таким образом наклон касательной к графику \(e^x - e^y = B\) в точке
\(x_0, y_0\) будет равен \(\dfrac{e^{x_0}}{e^{y_0}}\).

Заметим, что произведение полученных коэффициентов наклона касательных в точке
\((x_0, y_0)\) равно \(-1\). Это значит, что касательные в этой точке
перпендикулярны, а значит перпендикулярны и исходные кривые.

\begin{figure}[!htbp]
  \centering
  \asyinclude{asy/02-03.asy}
\end{figure}

\bigskip

Рассмотрим векторную линию поля \(e^{-x} + e^{-y} = e^2 + e\). Выберем две
точки, лежащие на этой линии: \(A(-2, -1)\) и \(B(-1, -2)\). Найдем работу поля
вдоль этой линии, используя найденный ранее потенциал:

\begin{equation*}
  \begin{split}
    W = \Phi \bigg\vert_A^B = \\
    (e^x - e^y) \bigg\vert_{(-2, -1)}^{(-1, -2)} = \\
    (e^{-1} - e^{-2}) - (e^{-2} - e^{-1}) = \\
    2e^{-1} - 2e^{-2} = \\
    \frac{2}{e^2}(e - 1)
  \end{split}
\end{equation*}
